%Proposal for the project work

\documentclass[sigplan,10pt,review]{acmart}
\settopmatter{printfolios=true,printccs=false,printacmref=false}

\usepackage{amsmath}
\usepackage{inputenc}

\title{Constraint driven Scheduling of Weak Memory C programs for HLS}

\begin{document}

    \section{Introduction}

        Scheduling in high level synthesis involves mapping programs to hardware clock cycles. 
        The resultant schedule determines how fast our program would run on such a synthesized hardware. 
        
        We focus on the scheduling problem when it comes to synthesizing hardware for weakly consistent concurrent programs.
        Previous work addressed this assuming we have infinite resources to synthesize such a hardware. 

        While in practice, this is not the case, we investigate the impact on scheduling given limited hardware resources to map concurrent threads. 
        We propose a pre-scheduling transformation that merges threads to fit in the resource constraint.
        We conjecture such a transformation would expose more efficient scheduling opportunities, previously invisible.
        As merging threads can be done in many combinations, this can result in different schedules.  
        We propose a global analysis to identify threads which can be merged to give us an efficient schedule among all the possible merges. 
        We test each(naive and global analysis) of these approaches in our code base, in comparison to previous work done assuming infinite resources. 

        %Simplified version
        Problem: Scheduling weak memory concurrent programs to clock cycles, given limited hardware resources to map concurrent threads.

        Objectives:
        \begin{itemize}
            \item Investigate the impact in scheduling.
            \item Propose to perform thread inlining transformation to improve scheduling.
            \item Applying global analysis to choose which threads to inline for efficient schedule.
        \end{itemize}


    \section{Related Work}

        Previous work made a compelling statement that we can synthesize hardware for relaxed atomics.
        They show how moving from coarse-grained (locks/unlocks) to fine-grained(atomics) concurrency introduces problems in incorrect scheduling of memory accesses.
        In particular, they link this problem to reordering of memory accesses.
        Their solution proposed adding more ordering constraints between such accesses based on the atomic access type. 
        This would enable much more efficient and correct scheduling in comparison to traditional lock based concurrency. 
        Their follow up work extends this to also schedule pipelined code efficiently.
        
        However, they assumed that to have infinite resources to synthesize such programs. 
        
        %Simplified version
        Previous work:
        1) Proposed mapping pthreads to hardware. 
        2) Proposed mapping fine grained atomics to hardware.
        3) Proposed correct scheduling for fine grained atomics being mapped to hardware. 
        4) Proposed global analysis for better scheduling fine grained atomics.

        Limitation:
        1) Assume infinite resources.
        2) Investigate only the aspect of reordering w.r.t. efficient scheduling. 

    \section{Methodology}

        \begin{itemize}
            \item Constraint the amount of hardware resource available to map threads. 
            \item Perform thread inlining, choosing randomly pairs of threads to merge, until resource constraint is met.
            \item Propose and apply a global analysis to better identify pairs of threads to inline.
            \item Observe the efficiency of resulting schedule.  
        \end{itemize}

        Benchmark:
        Existing lock free implementation of Message Passing and Buffers.
        Handwritten code (to expose an efficient thread inlining pair(s) exist).

        Simulator:
        Coursework setup augmented with required code to handle fine grained atomics. 
        Print schedule after scheduling phase. 

        Code to write:
        Add new AST node for Thread. 
        Add new "type" parameter for each memory access (read/write). 
        Augment previous work's ordering constraints to reflect their method of scheduling.
        Add a transformation step that merges two AST thread nodes. 
        Add a pre-analysis phase that gives out pairs of threads to merge that will lead to efficient scheduling.


    \section{Timeline}

        \begin{itemize}
            \item Week 1,2: Add AST node Thread and write sample programs to test memory accesses.
            \item Week 3,4: Add types for each memory access, test benchmark program naive scheduling (incorrect one)
            \item Week 5: Augment previous work addition to scheduling constraints, test benchmark programs. 
            \item Week 6: Augment thread resource constraint and merging transformation, test benchmark program schedules.
            \item Week 7,8: Add algorithm that takes program and gives out set of threads to inline, test benchmark program schedules.
            \item Week 9: Report results.
        \end{itemize}



\end{document}