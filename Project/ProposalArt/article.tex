%Proposal for the project work

\documentclass[sigplan,10pt,review]{acmart}
%\settopmatter{printfolios=true,printccs=false,printacmref=false}



\usepackage{amsmath}
\usepackage{inputenc}

\setcopyright{acmcopyright}
\copyrightyear{2022}
\acmYear{2022}
\acmDOI{}


\acmConference[COMP 764]{XYZ}{Winter 2022}{Montreal, Quebec, Canada}

\acmBooktitle{Yay book} 
\acmPrice{Free}
\acmISBN{}



\title{Constraint driven Scheduling of fine-grained C concurrency for Reconfigurable Hardware}
\author{Akshay Gopalakrishnan}
\affiliation{McGill University}
\email{akshay.akshay@mail.mcgill.ca}
\date{today}

\begin{document}

    \maketitle

    \section{Introduction}

        %What is Reconfigurable Hardware
        %What is HLS
        High Level Synthesis(HLS) is a process of mapping behavioral(C-like programs) models of hardware to low level(HDL-like programs) ones; which are then synthesized as specialized hardwares(eg: FPGA, ASIC).
        HLS tools are used widespread in recent times due to the high demand for such specialized hardware than have better throughput and energy consumption for specific intensive computations(eg: in Machine Learning).  
        
        The past decades have also progressed in better utilization of multicore machines for several concurrent applications. 
        Specifically, the use of fine grained concurrent features (eg: Load/Store buffers, speculation,etc.); whose descriptions are given by relaxed memory consistency models.

        Our focus is in scheduling of such fine grained concurrent programs that are synthesized to hardware. 
        Scheduling is a phase in HLS wherein programs (models) are mapped to hardware clock cycles. 
        The schedule determines how many clock cycles would it take for a computation to be done on the hardware synthesized. 
        
        %Scheduling for Concurrent Programs
        We specifically look at the impact on scheduling such programs in C given a constraint on number of hardware resources available to map concurrent processes(threads).
        Our contribution is three fold:
        \begin{itemize}
            \item Investigate the impact in scheduling such programs given thread resource constraint.
            \item Implement pre-scheduling optimization to improve the overall schedule.
            \item Add a global-analysis to improve the optimization to get a more efficient schedule.
        \end{itemize}


    \section{Related Work}

        \cite{DBLP:conf/fpt/ChoiBA13} propose a method to map software threads to parallel hardware for FPGAs. 
        They do this for programs that utilize pthreads and OpenMP constructs.
        However, their focus was mainly on lock-based concurrency in C. 
        However, they do note in their experiments that having a constraint on resources to map threads does effect the schedule(more cycles). 

        \cite{DBLP:conf/fpga/RamanathanFWC17} investigate the synthesis of concurrent programs utilizing fine grained concurrency elements of C.
        They show that on using atomics primitives provided by C, incorrect schedules can be obtained.
        They remedy this by adding additional ordering constraints among memory accesses.
        \cite{DBLP:journals/tc/RamanathanWC18} is a follow up to this work implementing pipelined scheduling for such programs.
        
        %Put optimization thing here
        \cite{DBLP:conf/lcpc/CongLPZ12} and \cite{DBLP:conf/fccm/HuangLCCXBA13} show the impact/effect of compiler optimizations on HLS for Reconfigurable hardware like FPGAs.
        Specifically, they show how scheduling(termed as latency) has a major impact due the optimization passes of the compiler, as well as the order in which we perform it. 
        But their work does not address this impact on concurrent programs(let alone those using fine grained Concurrency) synthesized to hardware. 

        While we do want to perform the same set of optimizations for concurrent models, we are constrained by the semantics of such programs.
        \cite{DBLP:conf/popl/VafeiadisBCMN15} shows that programs utilizing fine grained concurrency of C do not get the benefit of some optimizations as they are rendered unsafe in a concurrent context. 
        \cite{DBLP:conf/fccm/RamanathanCW18} try to identify which memory accesses can be safely reordered in a weakly consistent C program targeted for hardware synthesis.
        Their goal was to improve the overall schedule of the program, and in turn the hardware that is synthesized for it. 
        \cite{DBLP:journals/tvlsi/RamanathanCW21} is a follow up to this work describing a global analysis to more aggressively perform safe reorderings of memory accesses for the same purpose.

        Previous works in the line of synthesizing weakly consistent C programs assume that every thread can be mapped to a unique hardware accelerator during synthesis. 
        They do not investigate the effect of constraining the number of available accelerators.
        And while reordering was proposed in this line to improve schedules, no investigation was done on whether any optimization can be done to improve schedules under the constraint of available hardware to map threads.

    \section{Methodology}

        In the presence is resource constraint for mapping threads to hardware, more than one thread's code will have to be run by the same hardware resource.
        This brings about an interleaving behavior between those threads, with an added overhead to switch from on thread's code to another.
        This ``context switching" will cost more hardware clock cycles; leading to an inefficient schedule.

        Sequentialisation is a compiler transformation that combines two or more threads by placing each other's code blocks in a sequential order. 
        \cite{DBLP:journals/pcs/MoiseenkoPK21} have recorded that such a transformation can be done safely for weakly consistent C programs, given the recent memory consistency model.
        We propose to augment this optimization in pre-scheduling stage of HLS with the added constraint on hardware resources to map threads.

        For our purpose, such a transformation  would save firstly, the context-switching overhead.
        Traditional lock based concurrency would only benefit from saving clock cycles from this overhead of context-switch when it comes to scheduling shared-memory accesses.
        But we note, from previous work's observation, that our transformation would benefit more if we have fine grained concurrent elements. 
        Merging two thread's code can expose more shared-memory accesses that are independent and have no ordering constraints. Thus, they can be scheduled in the same clock cycle. 
        Additionally, taking motivation from previous work, it would also benefit to identify a global analysis to better perform this optimization. 
        
        For our contribution, we use our existing code base which synthesizes a subset of C like sequential program to VHDL language. 
        We extend our code base to incorporate the notion of threads, followed by augmenting it with weakly consistent memory accesses from previous work. 
        We then place a constraint on the resource count for mapping threads and observe the schedule that results due to previous work. 
        We then augment our naive version of transformation that performs sequentialisation before scheduling.
        We also then add a pre-analysis step for this transformation that results in a more efficient merging of threads which result in a better schedule.  
        We analyze the impact of our two approaches to previous work on the resultant schedule(s) obtained.
        For this, we utilize the benchmarks used in previous work, along with adding our own custom made programs which can expose the importance of the pre-analysis step.  
        In summary, we do the following:
        \begin{itemize}
            \item Encode previous work's contribution to our existing code base.
            \item Constrain the amount of hardware resources available to map threads. 
            \item Augment sequentialisation transformation, that chooses random pairs of threads to merge, until we can uniquely map each thread to a resource.
            \item Add a global-analysis step to better identify pairs of threads to sequentialise.
            \item Observe the outcome of the resulting schedule on lock free algorithms of message-passing and producer-consumer problems.
            \item Observe the outcome of the resulting schedule on custom-written lock free programs.  
        \end{itemize}
        
        Note that our current contributions are only restricted upto the scheduling stage on HLS.
        In addition, we consider our programs to not have loops/conditionals.
        As part of our future work, we would like to firstly extend this to add features of loops and conditionals, followed by incorporating our work in existing HLS tools like LegUp, that can synthesize concurrent C models to hardware.

        In our current work, we will only show how this transformation in unison with reordering(from previous work) improves the schedule.
        But it is important to note that several other optimizations in unison can further give us a better scheduled circuit.
        We leave this analysis as part of future work.

    \section{Timeline}

        \begin{itemize}
            \item Week 1: Add AST node Thread and write sample programs to test scheduling of memory accesses.
            \item Week 2,3: Add memory ordering types for each memory access, test benchmark program naive scheduling (incorrect one).
            \item Week 5: Augment previous work addition to scheduling constraints, test benchmark programs. 
            \item Week 6: Augment thread resource constraint and merging transformation, test benchmark program schedules.
            \item Week 7,8: Add global analysis to improve sequentialisation, test benchmark program schedules.
            \item Week 9: Report results.
        \end{itemize}


    \bibliographystyle{ACM-Reference-Format}
    \bibliography{ref}

\end{document}