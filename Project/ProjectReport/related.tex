%Related work

\section{Related Work}

    %Start with HLS tools
    Several HLS tools are available today; some of the notables being Quartus and LegUp. 
    LegUp allows their users to insert behavioral models of hardware in C language. 
    Several works were done further to extend LegUp in synthesizing concurrent C programs to hardware.  
    
    \cite{DBLP:conf/fpt/ChoiBA13} propose a method to map software threads to parallel hardware for FPGAs. 
    They do this for programs that utilize pthreads and OpenMP constructs.
    However, their focus was mainly on lock-based concurrency in C. 
    They do note in their experiments that having a constraint on resources to map threads does effect the schedule(more cycles). 
    Their contributions were integrated into LegUp.

    \cite{DBLP:conf/fpga/RamanathanFWC17} investigate the synthesis of concurrent programs utilizing fine grained concurrency elements of C.
    They show that on using atomics primitives provided by C, incorrect schedules can be obtained.
    They remedy this by adding additional ordering constraints among memory accesses.
    They test their approach by extending LegUp to tackle fine grained concurrent constructs of C.
    \cite{DBLP:journals/tc/RamanathanWC18} is a follow up to this work implementing pipelined scheduling for such programs.
        
    %Put optimization thing here
    When it comes to HLS tools, one major advantage is the optimizations we can do at the behavioral model itself. 
    These optimizations, have also been proven to significantly improve the resultant hardware synthesized.
    \cite{DBLP:conf/lcpc/CongLPZ12} and \cite{DBLP:conf/fccm/HuangLCCXBA13} show the impact/effect of compiler optimizations on HLS for Reconfigurable hardware like FPGAs.
    Specifically, they show how scheduling(termed as latency) has a major impact due the optimization passes of the compiler, as well as the order in which we perform it. 
    But their work does not address this impact on concurrent programs(let alone those using fine grained Concurrency) synthesized to hardware. 

    While we do want to perform aggressive optimizations for concurrent behavioral models, we are constrained by the semantics of such programs.
    \cite{DBLP:conf/popl/VafeiadisBCMN15} shows that programs utilizing fine grained concurrency of C do not get the benefit of some optimizations as they are rendered unsafe in a concurrent context. 
    \cite{DBLP:conf/fccm/RamanathanCW18} try to identify which memory accesses can be safely reordered in a weakly consistent C program targeted for hardware synthesis.
    Their goal was to improve the overall schedule of the program, and in turn the hardware that is synthesized for it. 
    \cite{DBLP:journals/tvlsi/RamanathanCW21} is a follow up to this work describing a global analysis to more aggressively perform safe reorderings of memory accesses for the same purpose.

    Previous works in the line of synthesizing weakly consistent C programs assume that every thread can be mapped to a unique hardware accelerator during synthesis. 
    They do not investigate the effect of constraining the number of available accelerators.
    While reordering was proposed in this line to improve schedules, no investigation was done on whether any optimization could have an impact on number of resources required for synthesis. 
    Moreover, minimization of resources was not a factor considered in this effort; as the primary goal was to get overall correct schedule; and doing optimizations to help lower the required clock cycles. 




    %Show why optimizations can be effective for better synthesis
    %Show work that investigates on synthesizing concurrent C programs to FPGA
    %Show work that investigates syntehsizes fine grained concurrent C programs 
    %Showcase the limitations for each of them
    